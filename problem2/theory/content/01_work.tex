\section*{Прикладная задача}
Дано $N$ независимых работ, для каждой работы задано время выполнения.
Требуется построить расписание выполнения работ без прерываний на $M$
процессорах. На расписании должно достигаться минимальное значение
\textit{критерия К2}.

\textbf{Критерий К2:} суммарное время ожидания (т.е. сумма, по всем работам в
расписании, времён завершения работ)
\section*{Формальная постановка задачи}
\textbf{Дано:}
    \begin{itemize}
        \item $N$ -- количество работ.
        \item $M$ -- количество процессоров.
        \item $P = \{p_i\}$ -- множество работ, где $p_i = \{N_i, W_i\}$ и
            $i = \overline{1, N}$. $N_i$ -- номер $i$-й работы, $W_i$ -- её
            время выполнения.
        \item $PU = \{m_j\}$ -- множество процессоров, где $m_j$ -- $j$-й
            процессор, и $j = \overline{1, M}$.
        \item $HP = (HP_B, HP_L)$ -- расписание, где $HP_B : P \to PU$ --
            привязка работ к процессору, $HP_L$ -- порядок выполнения работ.
    \end{itemize}
    При этом выполняются следующие условия, которые позволяют считать
    расписание корректным:
    \begin{itemize}
        \item $(p_i, p_j) \in HP^*_L \Rightarrow (p_j, p_i) \notin HP^*_L$, где
            $HP^*_L$ -- транзитивное замыкание $HP_L$.
            Т.е. расписание ациклично.
        \item $\forall p_i, i = \overline{1, N} \ \exists! m_k :
            HP_B(p_i) = m_k$. Т.е. каждая работа должна быть распределена на
            какой-либо процессор и только на один процессор.
    \end{itemize}

\textbf{Требуется:}\\
    Построить расписание $HP$.

\textbf{Минимизируемый критерий:}\\
    Определим $t_i$, как время завершения $i$-й работы. Тогда
    $$t_i = \sum\limits_{j} W_j + W_i,$$
    где
    $$j : HP_B(p_j) = HP_B(p_i), (p_j, p_i) \in HP^*_L.$$
    Определим $T$, как
    $$T = \sum\limits^{N}_{i = 0} t_i.$$
    Тогда минимизируемым критерием является
    $$\min\limits_{HP \in HP^*} T,$$
    где $HP^*$ -- множество корректных расписаний.


% \section{Ход работы}

% \textit{Ниже написан полнейший БРЕД (точнее компиляция лабораторных), чтобы показать, как делать то или иное действие в латехе и оверлифе. Все совпадения случайны.}

% Дана система с произвольной задержкой

% $\dot{x} = Ax(t) + A_1 x (t-h), $
% $A =
% \begin{bmatrix}
% 1 &  0\\
% 4 & 3\\
% \end{bmatrix},
% A_1 =
% \begin{bmatrix}
% -2 & 1 \\
% -2 & -6\\
% \end{bmatrix}$

% $\begin{bmatrix}
% \Phi & P - P_2^T (A+A_1)^T P_3 & -h P_2^T A_1\\
% * & -P_3 - P_3^T + hR & -h P_3^T A_1\\
% * & * & -hR\\
% \end{bmatrix} < 0,
% $

% $\Phi = P_2^T (A+A_1) + (A+A_1)^T P_2, P > 0, R > 0$

% Здесь $P_2$ и $P_3$ – произвольные матрицы. В результате решения матричного неравенства выше в MATLAB получена максимальная задержка $h = 0.18$, при которой система является устойчивой.

% Схема моделирования была собрана. Опытным путём удалось установить значение $K_{OC} = 2.5$ - максимальное допустимое, то есть значение при котором система устойчива (это граница устойчивости).

% \begin{figure}[H]
%     \centering
% \includegraphics[width=1.\linewidth,center]{scheme_1.png}
%     \caption{Схема моделирования}
%     \label{fig:my_label}
% \end{figure}

% % место для вывода про экстраполятор

% \subsection{Что-то там про колебательность}

% Колебательность процесса при фиксированном периоде квантования $T$ зависит от $K_{OC}$ следующим образом: чем больше $K_{OC}$ - тем больше система будет колебаться, пока не дойдёт до границы устойчивости, на которой она будет вести себя как чисто колебательная система без стремления к нулю.

% На графиках ниже можно видеть, как переходный процесс становится всё более колебательным и колебательным.

% Пусть степень устойчивости $\alpha = 0.5$. Тогда

% \begin{enumerate}
%     \item При $\mu = 3, \; \sigma(A+BK) = \{  -1.3192 \pm 5.2207i,  -0.9915, -1\}$
%     \item При $\mu = 1, \; \sigma(A+BK) = \{  -1.1340 \pm 5.5547i,  -0.9361, -1\}$
%     \item При $\mu = 0.4, \; \sigma(A+BK) = \{  -0.9402 \pm 5.8582i,  -0.7433, -1\}$
% \end{enumerate}


% \subsubsection{Графики}

% \begin{figure}[H]
%     \centering
% \includegraphics[width=1.\linewidth,center]{plot_y_2.png}
%     \caption{График $y(t)$ при $K_{OC} = 2$}
%     \label{fig:my_label}
% \end{figure}


% \begin{figure}[H]
%     \centering
% \includegraphics[width=1.\linewidth,center]{plot_y_23.png}
%     \caption{График $y(t)$ при $K_{OC} = 2.3$}
%     \label{fig:my_label}
% \end{figure}


% \begin{figure}[H]
%     \centering
% \includegraphics[width=1.\linewidth,center]{plot_y_25.png}
%     \caption{График $y(t)$ при $K_{OC} = 2.5$}
%     \label{fig:my_label}
% \end{figure}

% \begin{figure}[!htbp]
% \centering
% \begin{subfigure}{.5\textwidth}
%   \centering
%   \includegraphics[width=0.8\linewidth]{download (16).png}
%   \caption{Зелёные}
%   \label{fig:sub1}
% \end{subfigure}%
% \begin{subfigure}{.5\textwidth}
%   \centering
%   \includegraphics[width=0.8\linewidth]{download (17).png}
%   \caption{Оранжевые}
%   \label{fig:sub2}
% \end{subfigure}
% \label{fig:test}
% \caption{Сегментированные области}
% \end{figure}

% Максимальная колебательность наблюдается при $K_{OC} = 2.5$

% \subsubsection{Код Python}

% \begin{lstlisting}[language=Python, caption=Импорт и обычная бинаризация]
% import cv2
% from google.colab.patches import cv2_imshow
% from matplotlib import pyplot as plt
% import numpy as np
% from math import *
% import skimage
% from skimage import data, io, filters

% I=cv2.imread("pic.jpg",cv2.IMREAD_GRAYSCALE)
% cv2_imshow(I)
% t=127
% ret,Inew=cv2.threshold(I,t,255,cv2.THRESH_BINARY)
% plt.imshow(Inew)
% \end{lstlisting}
